\section*{FIFO Priority Exploration}

Consider a graph with 20--30 nodes.
Provide the array of active nodes in FIFO order and the set of explored nodes at each iteration.
For first 3--4 iterations, explain how the array is being populated and which nodes are being removed.

\begin{figure}[htbp]
      \centering
      \includegraphics[width=0.8\textwidth]{../Files/fifo.png}
      \vspace*{-2em}
      \caption{
            FIFO Priority Exploration
      }\label{fig:q1}
\end{figure}

Give the steps for searching from initial state \(I\) to  goal state \(G\) using FIFO priority function.

\subsection*{Solution}

Starting BFS from node 0.

\subsubsection*{Iteration 1}
Active queue: [0] \\
Explored set: \{\} \\
Pop node 0. \\
Neighbors of 0: 1, 7. \\
Add 1 and 7 to queue. \\
New queue: [1, 7] \\
Explored: \{0\}

\subsubsection*{Iteration 2}
Active queue: [1, 7] \\
Explored set: \{0\} \\
Pop node 1. \\
Neighbors of 1: 0, 2, 12, 16, 18. \\
0 already explored. Add 2, 12, 16, 18 to queue. \\
New queue: [7, 2, 12, 16, 18] \\
Explored: \{0, 1\}

\subsubsection*{Iteration 3}
Active queue: [7, 2, 12, 16, 18] \\
Explored set: \{0, 1\} \\
Pop node 7. \\
Neighbors of 7: 0, 3, 4, 12, 19. \\
0 and 12 already seen. Add 3, 4, 19 to queue. \\
New queue: [2, 12, 16, 18, 3, 4, 19] \\
Explored: \{0, 1, 7\}

\subsubsection*{Iteration 4}
Active queue: [2, 12, 16, 18, 3, 4, 19] \\
Explored set: \{0, 1, 7\} \\
Pop node 2. \\
Neighbors of 2: 1, 9, 10, 11. \\
1 already explored. Add 9, 10, 11 to queue. \\
New queue: [12, 16, 18, 3, 4, 19, 9, 10, 11] \\
Explored: \{0, 1, 7, 2\}
