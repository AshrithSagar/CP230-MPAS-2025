\section*{Question 1}

Consider a geometry,
\vspace*{-1.5em}
\begin{figure}[htbp]
      \centering
      \includegraphics[width=0.7\textwidth]{../../../Notebook/TeX/2025-03-05/4}
\end{figure}
\vspace*{-1.5em}
Assume values for \( \gamma = \frac{V_A}{V_B}, R_0, \cancel{\theta_0}, \alpha_{B0} \).
\begin{enumerate}[label= (\alph*), noitemsep, topsep=0pt]
      \item Find the collision cone.
            Find \( \alpha_A \)
      \item Take different values of \( \gamma \).
            Are there values of \( \gamma \) for which
            \begin{enumerate}[label= (\alph*), noitemsep, topsep=0pt]
                  \item CC does not exist?
                  \item There are more than one CC.
                        How many are possible?
            \end{enumerate}
            \vspace*{-1.5em}
            \begin{figure}[htbp]
                  \centering
                  \includegraphics[width=0.5\textwidth]{../../../Notebook/TeX/2025-03-05/5}
            \end{figure}
\end{enumerate}

\vspace*{-3em}
\subsection*{Solution}

Given the geometry as above, and in addition, \( B \) has a radius \( R_B \).
We have the following equations, for some \( \alpha_A \),
\begin{align*}
      V_R      & = V_B \cos(\alpha_B - \theta) - V_A \cos(\alpha_A - \theta) \\
      V_\theta & = V_B \sin(\alpha_B - \theta) - V_A \sin(\alpha_A - \theta)
\end{align*}
where \( V_R \) is the relative velocity along the line of separation and \( V_\theta \) is the relative velocity perpendicular to the line of separation.
The collision cone is the set of all possible directions of \( V_A \) such that \( A \) can collide with \( B \).

We have the following inequality, for collision to happen
\begin{equation*}
      \boxed{ (R_0^2 - R_B^2) {(V_B \sin (\alpha_B - \theta_0) - V_A \sin (\alpha_A - \theta_0))}^2 \leq R_B^2 {(V_B \cos (\alpha_B - \theta_0) - V_A \cos (\alpha_A - \theta_0))}^2 }
\end{equation*}

\clearpage
The collision cone parameters in the simulation were set to \( V_A = 10, V_B = 5, R_0 = 10, \theta = \frac{\pi}{6} = 30^\circ, \alpha_B = \frac{2\pi}{3} = 120^\circ, R_B = 2.0 \).

\vspace*{-1em}
\subsubsection*{Collision cone}

Output:

Minimum alpha\_A : 0.83 radians = 47.56 degrees

Maximum alpha\_A : 3.35 radians = 191.94 degrees

\vspace*{-1em}
\subsubsection*{Different values of \( \gamma \)}

\textbf{(a) No collision cone exists}

\vspace*{-2em}
\begin{figure}[htbp]
      \centering
      \includegraphics[width=0.6\textwidth]{./figures/f1.png}
\end{figure}

\vspace*{-1.5em}
As we can see from the figure, for \( 0 < \gamma < 1 \), the collision cone does not exist.

. \\
\textbf{(b) Multiple collision cones}

\vspace*{-1em}
\begin{figure}[htbp]
      \centering
      \includegraphics[width=0.6\textwidth]{./figures/f2.png}
\end{figure}

\vspace*{-1.5em}
Depending on how \( \gamma \) is defined, the multiple possibilities arise when \( V_A,  V_B \), and their negative values are considered.
At most, there can be 2 collision cones.

The above figure shows the cases when \( \gamma < 0 \).

\clearpage
\subsection*{Solution (incorrect)}

This is included here just for the sake of completeness.

\subsubsection*{Collision cones}

The collision cone is the set of all possible directions of \( V_A \) such that \( A \) can collide with \( B \).
The relative velocity along the line of separation is given by
\begin{align*}
      V_{R}
       & =
      V_B \cos (\alpha_B - \theta) - V_A \cos (\alpha_A - \theta)
      \\
      \implies
      \frac{V_R}{V_B}
       & =
      \cos (\alpha_B - \theta) - \gamma \cos (\alpha_A - \theta)
\end{align*}
where \( \alpha_A \) is the angle between \( V_A \) and the reference.
Collisions occur when \( V_R < 0 \).
\begin{align*}
      \implies &
      \cos (\alpha_B - \theta) - \gamma \cos (\alpha_A - \theta) < 0
      \\ \implies &
      \cos (\alpha_B - \theta) < \gamma \cos (\alpha_A - \theta)
\end{align*}
\begin{align*}
      \implies &
      \cos (\alpha_A - \theta) > \frac{1}{\gamma} \cos (\alpha_B - \theta)
      \\ \implies &
      \vert \alpha_A - \theta \vert < \cos^{-1} \left( \frac{1}{\gamma} \cos (\alpha_B - \theta) \right)
\end{align*}
since \( \cos(\cdot) \) is a decreasing function in \( [0, \pi] \). Thereby, defining \( \alpha_{A -} \) and \( \alpha_{A +} \) as
\begin{equation*}
      \boxed{
            \begin{aligned}
                  \alpha_{A -}
                   & =
                  \theta - \cos^{-1} \left( \frac{1}{\gamma} \cos (\alpha_B - \theta) \right)
                  \\
                  \alpha_{A +}
                   & =
                  \theta + \cos^{-1} \left( \frac{1}{\gamma} \cos (\alpha_B - \theta) \right)
            \end{aligned}
      }
\end{equation*}
the collision cone is given by
\begin{equation*}
      \boxed{
            \bar \alpha_A = \{ \alpha_A \mid \alpha_{A -} < \alpha_A < \alpha_{A +} \}
      }
\end{equation*}

\subsubsection*{Different values of \( \gamma \)}

\textbf{(a) No collision cone exists}

If \( \gamma < 1 \), then \( \frac{1}{\gamma} > 1 \).
Thus, \( \cos^{-1} \left( \frac{1}{\gamma} \cos (\alpha_B - \theta) \right) \) is not defined.
Therefore, no collision cone exists.
Intuitively, we can see that if \( \gamma < 1 \iff V_A < V_B \), thereby \( A \) is slower than \( B \) and cannot catch up to \( B \).

Therefore, the \underline{values of \( \gamma \) for which no collision cone exists are \( \gamma < 1 \)}.
(Incorrect; This seems to be not neccessarily the case.) \\
\textbf{(b) Multiple collision cones}

If \( \gamma > 1 \), then \( \frac{1}{\gamma} < 1 \).
Thus, \( \cos^{-1} \left( \frac{1}{\gamma} \cos (\alpha_B - \theta) \right) \) is defined.
