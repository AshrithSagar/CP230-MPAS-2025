\section*{Question 1}

Consider two robots of radii \( R_1 \) and \( R_2 \), moving in straight lines such that initially the distance between them is shrinking.
\begin{enumerate}[label= (\alph*), noitemsep, topsep=0pt]
    \item Find the distance between them at the point of closest approach when they do not collide, and the time at which this happens.
    \item Find out the times at which they first collide and then the time at which they get out of collision, assume the robots to be permeable.
\end{enumerate}
In both cases, choose the initial conditions accordingly. Choose the values of all parameters such as radii, initial positions, directions, speeds, etc.

\vspace*{-1em}
\subsection*{Solution (a): No collision case}

\vspace*{-1em}
\subsubsection*{Approach-1: (Lengthy expansions)}

\vspace*{-3em}
\begin{figure}[htbp]
    \centering
    \includegraphics[width=0.7\textwidth]{figures/q1-1}
    \vspace*{-2em}
    \caption{
        Collision diagram
    }\label{fig:q1-1}
\end{figure}

Let the initial positions of the robots be \( (x_1, y_1) \) and \( (x_2, y_2) \) respectively, and their velocities be \( (v_{x1}, v_{y1}) \) and \( (v_{x2}, v_{y2}) \) respectively.
At any time \( t \), we can see that, provided the robots don't collide, the position of the robots follow the equations
\begin{align*}
    x_1(t) & = x_1 + v_{x1}t \quad \text{and} \quad y_1(t) = y_1 + v_{y1}t \\
    x_2(t) & = x_2 + v_{x2}t \quad \text{and} \quad y_2(t) = y_2 + v_{y2}t
\end{align*}

Since the robots don't collide, we have the condition that the distance between the robots is always greater than the sum of their radii, i.e.,
\begin{equation*}
    \boxed{
        d(t) > R_1 + R_2, \quad \forall t \geq 0
    }
\end{equation*}

The distance between the robots at time \( t \) follows
\begin{align*}
    d^2(t)
     & =
    (x_2(t) - x_1(t))^2 + (y_2(t) - y_1(t))^2
    \\ & =
    \big((x_2 + v_{x2}t) - (x_1 + v_{x1}t)\big)^2 + \big((y_2 + v_{y2}t) - (y_1 + v_{y1}t)\big)^2
    \\ & =
    \big((x_2 - x_1) + (v_{x2} - v_{x1})t\big)^2 + \big((y_2 - y_1) + (v_{y2} - v_{y1})t\big)^2
\end{align*}

Differentiating both sides with respect to \( t \) and setting the derivative to zero to find the minimum distance, we get the \underline{time of closest approach \( t_c \)} as
\begin{align*}
    \frac{d}{dt}d^2(t)
     & =
    2\big((x_2 - x_1) + (v_{x2} - v_{x1})t\big)(v_{x2} - v_{x1}) + 2\big((y_2 - y_1) + (v_{y2} - v_{y1})t\big)(v_{y2} - v_{y1})
    \\ \implies
    0
     & =
    2\big((x_2 - x_1)(v_{x2} - v_{x1}) + (y_2 - y_1)(v_{y2} - v_{y1})\big) + 2\big((v_{x2} - v_{x1})^2 + (v_{y2} - v_{y1})^2\big) t_c
\end{align*}
\begin{equation*}
    \implies
    \boxed{
        t_c
        =
        -\frac{(x_2 - x_1)(v_{x2} - v_{x1}) + (y_2 - y_1)(v_{y2} - v_{y1})}{(v_{x2} - v_{x1})^2 + (v_{y2} - v_{y1})^2}
    }
\end{equation*}
or in vector form
\begin{equation*}
    t_c
    =
    -\frac{\vec{r} \cdot \vec{v}}{\vec{v} \cdot \vec{v}}
\end{equation*}
where \( \vec{r} = \begin{bmatrix} x_2 - x_1 \\ y_2 - y_1 \end{bmatrix} \) and \( \vec{v} = \begin{bmatrix} v_{x2} - v_{x1} \\ v_{y2} - v_{y1} \end{bmatrix} \).

The corresponding \underline{distance at the point of closest approach \( d_c \)} follows
\begin{align*}
    d_c^2
    = d^2(t_c)
     & =
    {\big((x_2 - x_1) + (v_{x2} - v_{x1})t_c\big)}^2 + {\big((y_2 - y_1) + (v_{y2} - v_{y1})t_c\big)}^2
    \\ & =
    (x_2 - x_1)^2 + (y_2 - y_1)^2
    \\ & \qquad
    + 2\big((x_2 - x_1)(v_{x2} - v_{x1}) + (y_2 - y_1)(v_{y2} - v_{y1})\big)t_c
    \\ & \qquad
    + \big((v_{x2} - v_{x1})^2 + (v_{y2} - v_{y1})^2\big)t_c^2
    \\ & =
    (x_2 - x_1)^2 + (y_2 - y_1)^2
    \\ & \qquad
    - 2\frac{{\big((x_2 - x_1)(v_{x2} - v_{x1}) + (y_2 - y_1)(v_{y2} - v_{y1})\big)}^2}{(v_{x2} - v_{x1})^2 + (v_{y2} - v_{y1})^2}
    \\ & \qquad
    + \frac{{\big((x_2 - x_1)(v_{x2} - v_{x1}) + (y_2 - y_1)(v_{y2} - v_{y1})\big)}^2}{(v_{x2} - v_{x1})^2 + (v_{y2} - v_{y1})^2}
    \\ & =
    (x_2 - x_1)^2 + (y_2 - y_1)^2
    \\ & \qquad
    - \frac{{\big((x_2 - x_1)(v_{x2} - v_{x1}) + (y_2 - y_1)(v_{y2} - v_{y1})\big)}^2}{(v_{x2} - v_{x1})^2 + (v_{y2} - v_{y1})^2}
\end{align*}
\begin{align*}
    \implies
    \big( (v_{x2} - v_{x1})^2 + (v_{y2} - v_{y1})^2 \big) d_c^2
     & =
    \big( (v_{x2} - v_{x1})^2 + (v_{y2} - v_{y1})^2 \big) \big( (x_2 - x_1)^2 + (y_2 - y_1)^2 \big)
    \\ & \qquad
    - \big( (x_2 - x_1)(v_{x2} - v_{x1}) + (y_2 - y_1)(v_{y2} - v_{y1}) \big)^2
    \\ & =
    \cancel{(v_{x2} - v_{x1})^2 (x_2 - x_1)^2} + \cancel{(v_{y2} - v_{y1})^2 (y_2 - y_1)^2}
    \\ & \qquad
    + (v_{x2} - v_{x1})^2 (y_2 - y_1)^2 + (v_{y2} - v_{y1})^2 (x_2 - x_1)^2
    \\ & \qquad
    - \cancel{(x_2 - x_1)^2 (v_{x2} - v_{x1})^2} - \cancel{(y_2 - y_1)^2 (v_{y2} - v_{y1})^2}
    \\ & \qquad
    - 2 (x_2 - x_1)(v_{x2} - v_{x1})(y_2 - y_1)(v_{y2} - v_{y1})
    \\ & =
    \big( (v_{x2} - v_{x1})(y_2 - y_1) - (v_{y2} - v_{y1})(x_2 - x_1) \big)^2
\end{align*}
\begin{equation*}
    \implies
    d_c^2
    =
    \frac{\big( (v_{x2} - v_{x1})(y_2 - y_1) - (v_{y2} - v_{y1})(x_2 - x_1) \big)^2}{(v_{x2} - v_{x1})^2 + (v_{y2} - v_{y1})^2}
\end{equation*}
\begin{equation*}
    \implies
    d(t_c) =
    \boxed{
        d_c
        =
        \frac{\big| (v_{x2} - v_{x1})(y_2 - y_1) - (v_{y2} - v_{y1})(x_2 - x_1) \big|}{\sqrt{(v_{x2} - v_{x1})^2 + (v_{y2} - v_{y1})^2}}
    }
\end{equation*}

\vspace*{-1em}
\subsubsection*{Approach-2: (Relative velocities)}

\vspace*{-3em}
\begin{figure}[htbp]
    \centering
    \includegraphics[width=0.7\textwidth]{figures/q1-2}
    \vspace*{-2em}
    \caption{
        Collision diagram
    }\label{fig:q1-2}
\end{figure}

Can consider \( v_r \) and \( v_\theta \) as the relative velocities of the robots in the radial and tangential directions respectively, and solve accordingly in this polar coordinate system.

\vspace*{-1em}
\subsection*{Solution (b): Collision case}

When the robots first collide, we can see that the distance between the robots at the time of collision is equal to the sum of their radii, i.e.,
\begin{equation*}
    \boxed{ d = R_1 + R_2 }
\end{equation*}

The distance between the robots at time \( t \) follows, as before
\begin{align*}
    d^2(t)
     & =
    \big((x_2 + v_{x2}t) - (x_1 + v_{x1}t)\big)^2 + \big((y_2 + v_{y2}t) - (y_1 + v_{y1}t)\big)^2
    \\ & =
    \big((x_2 - x_1) + (v_{x2} - v_{x1})t\big)^2 + \big((y_2 - y_1) + (v_{y2} - v_{y1})t\big)^2
\end{align*}
Now, we have
\begin{align*}
    (R_1 + R_2)^2
     & =
    \big((x_2 - x_1) + (v_{x2} - v_{x1})t\big)^2 + \big((y_2 - y_1) + (v_{y2} - v_{y1})t\big)^2
    \\ & =
    (x_2 - x_1)^2 + (y_2 - y_1)^2
    \\ & \qquad
    + 2\big((x_2 - x_1)(v_{x2} - v_{x1}) + (y_2 - y_1)(v_{y2} - v_{y1})\big)t
    \\ & \qquad
    + \big((v_{x2} - v_{x1})^2 + (v_{y2} - v_{y1})^2\big)t^2
\end{align*}
which is a quadratic equation in \( t \), which can be solved to find the times at which the robots first collide and then get out of collision.

The determinant of the quadratic equation is
\begin{align*}
    \Delta
     & =
    4\big((x_2 - x_1)(v_{x2} - v_{x1}) + (y_2 - y_1)(v_{y2} - v_{y1})\big)^2
    \\ & \qquad
    - 4\big((v_{x2} - v_{x1})^2 + (v_{y2} - v_{y1})^2\big)\big((x_2 - x_1)^2 + (y_2 - y_1)^2 - (R_1 + R_2)^2\big)
\end{align*}
For real collisions to occur, we need \( \Delta \geq 0 \).
The two times at which the robots collide are given by
\begin{align*}
     &
    \boxed{
        t_{\text{first collision}}
        =
        \frac{-2\big((x_2 - x_1)(v_{x2} - v_{x1}) + (y_2 - y_1)(v_{y2} - v_{y1})\big) - \sqrt{\Delta}}{2\big((v_{x2} - v_{x1})^2 + (v_{y2} - v_{y1})^2\big)}
    }
    \\ &
    \boxed{
        t_{\text{get out of collision}}
        =
        \frac{-2\big((x_2 - x_1)(v_{x2} - v_{x1}) + (y_2 - y_1)(v_{y2} - v_{y1})\big) + \sqrt{\Delta}}{2\big((v_{x2} - v_{x1})^2 + (v_{y2} - v_{y1})^2\big)}
    }
\end{align*}
