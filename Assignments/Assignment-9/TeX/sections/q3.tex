\section*{Question 3}

\textbf{Transformed primitive}

Let \( \mathcal{A} \) be a unit disc, centered at origin, and \( \mathcal{W} = \mathbb{R}^2 \).
Assume that \( \mathcal{A} \) is represented by a single, algebraic primitive, \( H = \{ (x, y) \mid x^2 + y^2 \leq 1 \} \).
Show that the transformed primitive is unchanged after any rotation is applied.

\subsection*{Solution}

We want to show that if we rotate the unit disk \( H = \{(x,y)\mid x^2+y^2\le1\} \) by any angle \( \theta \), the set of points remains the same. Equivalently, if we apply the rotation map
\[
    R_\theta(x,y)
    =\begin{pmatrix}
        \cos\theta & -\sin\theta \\[6pt]
        \sin\theta & \cos\theta
    \end{pmatrix}
    \begin{pmatrix}x\\y\end{pmatrix},
\]
then
\[
    R_\theta(H)
    =\{\,R_\theta(x,y)\mid x^2+y^2\le1\}
    \;=\;H.
\]
which is what we want to show.

Let \( (x,y)\in H \); then
\[
    x^2+y^2\;\le1.
\]
Apply the rotation \( R_\theta \), giving
\[
    (x',y') \;=\; R_\theta(x,y)
    \;=\;
    \bigl(x\cos\theta - y\sin\theta,\;x\sin\theta + y\cos\theta\bigr).
\]
Compute the squared norm of \( (x',y') \):
\[
    \begin{aligned}
        {x'}^2 + {y'}^2
         & = (x\cos\theta - y\sin\theta)^2 \;+\;(x\sin\theta + y\cos\theta)^2 \\[6pt]
         & = x^2\cos^2\theta - 2xy\cos\theta\,\sin\theta + y^2\sin^2\theta
        \;+\; x^2\sin^2\theta + 2xy\sin\theta\,\cos\theta + y^2\cos^2\theta   \\[6pt]
         & = x^2(\cos^2\theta + \sin^2\theta)
        + y^2(\sin^2\theta + \cos^2\theta)
        + \bigl(-2xy\cos\theta\,\sin\theta + 2xy\sin\theta\,\cos\theta\bigr)  \\[6pt]
         & = x^2 + y^2 + 0
        \;=\; x^2 + y^2.
    \end{aligned}
\]
Since \( x^2+y^2\le1 \), it follows that \( {x'}^2+{y'}^2\le1 \). Hence
\[
    R_\theta(x,y)\in H,
\]
so \( R_\theta(H)\subseteq H \).

Conversely, given any \( (x',y') \) with \( {x'}^2+{y'}^2\le1 \), we can apply the inverse rotation \( R_{-\theta} \) to find \( (x,y)=R_{-\theta}(x',y') \). By the same computation,
\[
    x^2 + y^2 \;=\; {x'}^2 + {y'}^2 \;\le1,
\]
so \( (x,y)\in H \) and \( (x',y')=R_\theta(x,y) \). Thus every point of \( H \) arises as the rotation of some point in \( H \), giving
\[
    H\subseteq R_\theta(H).
\]

Combining the two inclusions,
\[
    R_\theta(H) \;=\; H,
\]
i.e.\ the \underline{transformed primitive is unchanged under any rotation}.
