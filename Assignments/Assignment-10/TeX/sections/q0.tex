\section*{Coordination algorithm}

\begin{enumerate}[noitemsep, leftmargin=0.5cm]
    \item Consider a \( 40 \times 40 \) grid map.

    \item Assume 6 obstacles, each covering about 100 cells.

    \item Assume 2 robots (assume that they start from the same cell. Sensor range is 6 cells).

    \item Find the route of each robot using coordination algorithm for 10 steps. \( P(d) \).
\end{enumerate}
In each iteration,
\begin{enumerate}[noitemsep]
    \item Shade and distinguish the frontier cells, the explored and the unexplored cells.

    \item Mention the utilities of the cells and how the utilities are being updated.

    \item Trace the robot path.
\end{enumerate}

\subsection*{Solution}

The attached code and simulation videos demonstrate the frontier cells, explored cells, unexplored cells, and the path traced by the robots.

A frontier cell is an explored free cell with at least one unknown neighbor.
Each frontier cell is initialised with a utility value of 1.0 in the beginning, i.e.
\begin{equation*}
    \mathcal{U}(f) = 1.0 \quad \text{for all frontier cells } f
\end{equation*}

The utility represents how desirable that cell is for exploration.
When a robot is assigned to a frontier cell, the utilities of other frontier cells are decreased based on their Manhattan distance \( d \) from the selected cell, as
\begin{equation*}
    \mathcal{U}(t) \gets \max\left(0, \mathcal{U}(t) - P(d)\right)
\end{equation*}
where \( P(d) \) is a a distance-based dropoff function, defined as
\begin{equation*}
    P(d) = \max\left(0, 1 - \frac{d}{R}\right)
\end{equation*}
where \( R \) is the sensor range of each robot.
Note that this function linearly decreases from 1.0 (when \( d = 0 \)) to 0.0 (when \( d \geq R \)).

Frontier cells close to the selected target lose more utility, which discourages other robots from duplicating effort.
On the other hand, distant frontiers retain utility, encouraging robots to spread out and cover new areas.
This helps in distributing robots across the grid, reducing overlap and improving efficiency.
The robots are greedily assigned to the frontier cell with the highest utility.

This approach ensures that the robots explore the environment efficiently, while also avoiding collisions and redundant exploration.
